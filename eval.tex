\section{Evaluation} \label{eval}
The section shows the evaluation of several graph computing systems:
GraphX, GraphLab (PowerGraph) and Giraph.
We conduct the evaluations of these systems in a cluster with 4 machines.
Each machine is equipped with a Xeon CPU with 24 hyper cores.
We evaluate PageRank with 2 datasets, the datasets we used are showed in
\tab{dataset}.

\begin{table}[tbh]
  \center
  \begin{tabular}{l|c|c}
    \textbf{Dataset Name} & \textbf{Vertex} & \textbf{Edges} \\
      \hline
      Youtube Network & 10k & 10m \\
      \hline
      Twitter Network & 1k & 1m
  \end{tabular}
  \caption{Dataset.}
  \label{tab:dataset}
\end{table}

Our evaluations mainly focus on performance of these system and evaluates
Gigraph, SparkX and PowerGraph.
Our evaluations mainly answer the following questions:
\begin{tightenum}

\item[\S\ref{subsec:time}:] How much time needed for each framework
to converge a PageRank?

\item[\S\ref{subsec:scalability}:] Are these frameworks scalable to
large graphs or large number of machines?

\item[\S\ref{subsec:technique}:] Are the techniques in each framework
effective to provide fast computations?
\end{tightenum}

% two graph (pagerank) two_dataset
\subsection{Computation Time} \label{subsec:time}
We first run the PageRank program with two datasets and compare the
time it takes to converge. \fig{time:a} and \fig{time:b}
shows the result. We used the asynchronous computation mode
for PowerGraph.

\begin{figure}[t]
  \includegraphics[width=.45\textwidth]{figures/time_a}
  \caption{}
  \label{fig:time:a}
\end{figure}

\begin{figure}[t]
  \includegraphics[width=.45\textwidth]{figures/time_b}
  \caption{}
  \label{fig:time:b}
\end{figure}

Among two datasets for PageRank, PowerGrapb has the best performance,
while GraphX is a little bit slower, Gigraph performs the worst.
PowerGraph is a C++ implementation with asynchronous execution, which
results in the best performance.

% differential numberof workers - one datasets
% differential number of graph - one datasets
\subsection{Scalability} \label{subsec:scalability}
\subsubsection{Scalability of Workers}
A good graph computing system should make better use of computing
resources. Therefore, when the number of computing resources increases,
the computing time should be shorten.

\begin{figure}[t]
  \includegraphics[width=.45\textwidth]{figures/scale_worker}
  \caption{}
  \label{fig:scale:worker}
\end{figure}

We run all systems with different numbers of workers using the same dataset,
and calculate the time for it to converge. \fig{scale:worker}
shows the result. As the result shows, it takes less time
when the number of workers increases. However, the time
does not decrease linearly as it takes much larger costs
for task scheduling and message transferring.

\subsubsection{Scalability of Datasets}
In practice, these graph computing system should be able to processing
graphs with large numbers of vertices and edges. The scale of graph
in practice is growing, and the graph size is billion and even trillion
in industrial applications. Therefore, it is highly desirable to
scale to large graphs.

\begin{figure}[t]
  \includegraphics[width=.45\textwidth]{figures/scale_dataset}
  \caption{}
  \label{}
\end{figure}

We run all systems with different sizes of datasets using
the same number of workers. \fig{scale:dataset} shows the time
for them to converge the result. As the result shows,
all of these system have good scalability. Some of them increase
sub-linearly while PowerGraph has the best scalability.


% Asyn mode of graphlab / comparism
\subsection{Technique Effectiveness} \label{subsec:technique}
In the section, we show the effectiveness of techniques in
different systems. For the short of time, we only evaluated
the effectiveness of asynchronous execution in this section.

\begin{table}
  \center
  \begin{tabular}{c|c}
    \hline
    \textbf{execution mode} & \textbf{time} \\
    \hline
    Synchronous Execution & 10s \\
    \hline
    Asynchronous Execution & 6s
  \end{tabular}
  \caption{}
  \label{tab:asyn}

\end{table}

\tab{asyn} shows the result of running with and without asynchronous
execution. The result shows that it takes less time with asynchronous
execution. With asynchronous execution, it takes less time for synchronization,
which results in much better performance.

% scalability (nodes and vertices) / time / effectiveness of several technique

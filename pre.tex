\section{Preliminary}
We first introduce some definition of graph computing.
Suppose there is a graph G = (V, E), $V$ and $E$ the vertex
and edges of the graph. Each vertex contains a state set $(s_1, s_2, ..., s_n)$,
and a set of neighboring vertices $v_{x1}, v_{x2}, ..., v{xn}$.
Each edge contains a state set (including the weight of edges) of
$(s_1, s_2, ..., s_n, w_i)$.

Initially, the graph G is stored in a database or files of HDFS. The system
first \textbf{load} data from storages. Each worker loads a subset of vertices
and edges of the graph G into memory according to the \textbf{partition scheme}
and initializes the values.
Then, each worker conducts iterative computations to
produce an output from vertex and edge values
according to the computing models. The output is used for updating
values in the second phase of computations. After that, the updated values
will be used in the new iteration.
When the computation converges, the system stops and the state of the graph
will be output to persistent storages.

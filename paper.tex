% This is sigproc-sp.tex -FILE FOR V2.6SP OF ACM_PROC_ARTICLE-SP.CLS
% OCTOBER 2002
%
% It is an example file showing how to use the 'acm_proc_article-sp.cls' V2.6SP
% LaTeX2e document class file for Conference Proceedings submissions.
% ----------------------------------------------------------------------------------------------------------------
% This .tex file (and associated .cls V2.6SP) *DOES NOT* produce:
%       1) The Permission Statement
%       2) The Conference (location) Info information
%       3) The Copyright Line with ACM data
%       4) Page numbering
%
%  However, both the CopyrightYear (default to 2002) and the ACM Copyright Data
% (default to X-XXXXX-XX-X/XX/XX) can still be over-ridden by whatever the author
% inserts into the source .tex file.
% e.g.
% \CopyrightYear{2003} will cause 2003 to appear in the copyright line.
% \crdata{0-12345-67-8/90/12} will cause 0-12345-67-8/90/12 to appear in the copyright line.
%
% ---------------------------------------------------------------------------------------------------------------
% It is an example which *does* use the .bib file (from which the .bbl file
% is produced).
% REMEMBER HOWEVER: After having produced the .bbl file,
% and prior to final submission,
% you need to 'insert'  your .bbl file into your source .tex file so as to provide
% ONE 'self-contained' source file.
%
% Questions regarding SIGS should be sent to
% Adrienne Griscti ---> griscti@acm.org
%
% Questions/suggestions regarding the guidelines, .tex and .cls files, etc. to
% Gerald Murray ---> murray@acm.org
%
% For tracking purposes - this is V2.6SP - OCTOBER 2002

\documentclass{acm_proc_article}

\begin{document}
%
% --- Author Metadata here ---
\conferenceinfo{COMP 9102 Project Report}{2017, CS Dept, HKU}
%\setpagenumber{50}
%\CopyrightYear{2002} % Allows default copyright year (2002) to be over-ridden - IF NEED BE.
%\crdata{0-12345-67-8/90/01}  % Allows default copyright data (X-XXXXX-XX-X/XX/XX) to be over-ridden.
% --- End of Author Metadata ---

\title{Fast and Optimized Graph Computing System}
%
% You need the command \numberofauthors to handle the "boxing"
% and alignment of the authors under the title, and to add
% a section for authors number 4 through n.

\author{
Cheng Wang\\
Department of Computer Science\\
University of Hong Kong\\
Pokfulam Road, Hong Kong\\
\texttt{cheng2@cs.hku.hk}
\and Jianyu Jiang\\
Department of Computer Science\\
University of Hong Kong \\
Pokfulam Road, Hong Kong \\
\texttt{jyjiang@cs.hku.hk}
}

\numberofauthors{2}

\newcommand{\eg}[0]{e.g.,}

\maketitle
\begin{abstract}
With the growing numbers of data, people desire a fast and scalable framework
for computations on them. However, general purpose computation
frameworks (\eg{} Spark~\cite{nsdi12:spark}, MapReduce~\cite{mapreduce})
are not suitable for graph computation algorithms such as PageRank and
ConnectedComponent. Theses frameworks generally generate prohibited amounts
of network traffic, so they are not scalable to datasets with billions or
even trillion number of records.

This paper revisits several graph computing systems and summarizes the similarity
and advantages of them. We also analyze how these systems make the efforts to
make computation scalable. We have done extensive evaluations on several graph
computing framework such as PowerGraph, Spark GraphX and ...
\end{abstract}


\section{Introduction}
General big-data computing framework (\eg{} Spark, MapReduce) provides
flexible and efficient programming primitives for most algorithms.
These frameworks split data into multiple partitions and run tasks on
these task independently. When necessary, shuffles happen to resolve data
dependencies.
However, these framework can result in a great amount of network traffic
for graph computing algorithms such as PageRank. Even worse, most realistic graphs
follows the power law, so joining two nodes may incur a one-to-all communications.

To tackle this problem, specified graph computing framework such as Pregel~\cite{sigmod10:pregel},
GraphLab~\cite{graphlab}, PowerGraph~\cite{powergraph} and GraphX~\cite{graphx}.

%ACKNOWLEDGMENTS are optional
\section{Acknowledgments}
This section is optional; it is a location for you
to acknowledge grants, funding, editing assistance and
what have you.  In the present case, for example, the
authors would like to thank Gerald Murray of ACM for
his help in codifying this \textit{Author's Guide}
and the \textbf{.cls} and \textbf{.tex} files that it describes.

%
% The following two commands are all you need in the
% initial runs of your .tex file to
% produce the bibliography for the citations in your paper.
\bibliographystyle{abbrv}
\bibliography{bib/biblio}  % sigproc-sp-csis8101.bib is the name of the Bibliography in this case
% You must have a proper ".bib" file
%  and remember to run:
% latex bibtex latex latex
% to resolve all references
%
% ACM needs 'a single self-contained file'!
%
%APPENDICES are optional
%\balancecolumns

% \appendix
% %Appendix A
% \section{Headings in Appendices}
% The rules about hierarchical headings discussed above for
% the body of the article are different in the appendices.
% In the \textbf{appendix} environment, the command
% \textbf{section} is used to
% indicate the start of each Appendix, with alphabetic order
% designation (i.e. the first is A, the second B, etc.) and
% a title (if you include one).  So, if you need
% hierarchical structure
% \textit{within} an Appendix, start with \textbf{subsection} as the
% highest level. Here is an outline of the body of this
% document in Appendix-appropriate form:
% \subsection{Introduction}
% \subsection{The Body of the Paper}
% \subsubsection{Type Changes and  Special Characters}
% \subsubsection{Math Equations}
% \paragraph{Inline (In-text) Equations}
% \paragraph{Display Equations}
% \subsubsection{Citations}
% \subsubsection{Tables}
% \subsubsection{Figures}
% \subsubsection{Theorem-like Constructs}
% \subsubsection*{A Caveat for the \TeX\ Expert}
% \subsection{Conclusions}
% \subsection{Acknowledgments}
% \subsection{Additional Authors}
% This section is inserted by \LaTeX; you do not insert it.
% You just add the names and information in the
% \texttt{{\char'134}additionalauthors} command at the start
% of the document.
% \subsection{References}
% Generated by bibtex from your ~.bib file.  Run latex,
% then bibtex, then latex twice (to resolve references)
% to create the ~.bbl file.  Insert that ~.bbl file into
% the .tex source file and comment out
% the command \texttt{{\char'134}thebibliography}.
% % This next section command marks the start of
% % Appendix B, and does not continue the present hierarchy
% \section{More Help for the Hardy}
% The acm\_proc\_article-sp document class file itself is chock-full of succinct
% and helpful comments.  If you consider yourself a moderately
% experienced to expert user of \LaTeX, you may find reading
% it useful but please remember not to change it.

% That's all folks!
\end{document}

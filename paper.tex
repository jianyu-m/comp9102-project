% This is sigproc-sp.tex -FILE FOR V2.6SP OF ACM_PROC_ARTICLE-SP.CLS
% OCTOBER 2002
%
% It is an example file showing how to use the 'acm_proc_article-sp.cls' V2.6SP
% LaTeX2e document class file for Conference Proceedings submissions.
% ----------------------------------------------------------------------------------------------------------------
% This .tex file (and associated .cls V2.6SP) *DOES NOT* produce:
%       1) The Permission Statement
%       2) The Conference (location) Info information
%       3) The Copyright Line with ACM data
%       4) Page numbering
%
%  However, both the CopyrightYear (default to 2002) and the ACM Copyright Data
% (default to X-XXXXX-XX-X/XX/XX) can still be over-ridden by whatever the author
% inserts into the source .tex file.
% e.g.
% \CopyrightYear{2003} will cause 2003 to appear in the copyright line.
% \crdata{0-12345-67-8/90/12} will cause 0-12345-67-8/90/12 to appear in the copyright line.
%
% ---------------------------------------------------------------------------------------------------------------
% It is an example which *does* use the .bib file (from which the .bbl file
% is produced).
% REMEMBER HOWEVER: After having produced the .bbl file,
% and prior to final submission,
% you need to 'insert'  your .bbl file into your source .tex file so as to provide
% ONE 'self-contained' source file.
%
% Questions regarding SIGS should be sent to
% Adrienne Griscti ---> griscti@acm.org
%
% Questions/suggestions regarding the guidelines, .tex and .cls files, etc. to
% Gerald Murray ---> murray@acm.org
%
% For tracking purposes - this is V2.6SP - OCTOBER 2002

% \documentclass[sigconf]{acmart}

\documentclass{acm_proc_article}

% \setcopyright{none}
% \acmDOI{none}

\newcommand{\eg}[0]{e.g.,}

\newcommand{\fig}[1]{Figure \ref{fig:#1}}
\newcommand{\tab}[1]{Table \ref{tab:#1}}
\newcommand{\chref}[1]{\S\ref{#1}}
% \renewcommand{\baselinestretch}{1.2}
\usepackage{tightenum}
\usepackage{hyperref}

\begin{document}
%
% --- Author Metadata here ---
\conferenceinfo{COMP 9102 Project Report}{2017, CS Dept, HKU}
%\setpagenumber{50}
%\CopyrightYear{2002} % Allows default copyright year (2002) to be over-ridden - IF NEED BE.
%\crdata{0-12345-67-8/90/01}  % Allows default copyright data (X-XXXXX-XX-X/XX/XX) to be over-ridden.
% --- End of Author Metadata ---

\title{Toward Practical Fast and Graph Computing Systems: \\A Survey on Graph Computing Systems}
%
% You need the command \numberofauthors to handle the "boxing"
% and alignment of the authors under the title, and to add
% a section for authors number 4 through n.

\author{
Cheng Wang\\
Department of Computer Science\\
University of Hong Kong\\
Pokfulam Road, Hong Kong\\
\texttt{cheng2@cs.hku.hk}
\and Jianyu Jiang\\
Department of Computer Science\\
University of Hong Kong \\
Pokfulam Road, Hong Kong \\
\texttt{jyjiang@cs.hku.hk}
}

\numberofauthors{2}

% \author{Cheng Wang}
% \affiliation{%
% \institution{The University of Hong Kong}
% \city{Hong Kong}
% \country{Hong Kong}}
% \email{cheng2@cs.hku.hk}
%
% \author{Jianyu Jiang}
% \affiliation{%
% \institution{The University of Hong Kong}
% \city{Hong Kong}
% \country{Hong Kong}}
% \email{jyjiang@cs.hku.hk}

\maketitle

\begin{abstract}
With the growing numbers of data, people desire a fast and scalable framework
for computations on them. However, general purpose computation
frameworks (\eg{} Spark, MapReduce)
are not suitable for graph computation algorithms such as PageRank and
ConnectedComponent. Theses frameworks generally generate prohibited amounts
of network traffic, so they are not scalable to datasets with billions or
even trillion number of records.

This paper revisits several graph computing systems and summarizes the similarity
and advantages of them. We first give a
brief introduction of some several characteristics of graph computing
systems and show how all these graph computing system make their design choice.
We also analyze how these systems make the efforts to
make computation scalable. We have done extensive evaluations on several graph
computing framework such as PowerGraph, Spark GraphX and Gigraph.
\end{abstract}



\section{Introduction}
General big-data computing frameworks (\eg{} Spark~\cite{spark}, MapReduce~\cite{mapreduce}) provide
flexible and efficient programming primitives for most algorithms.
These frameworks split data into multiple partitions and run tasks on
these task independently. When necessary, shuffles happen to resolve data
dependencies.
However, these framework can result in a great amount of network traffic
for graph computing algorithms such as PageRank. Even worse, most realistic graphs
follows the power law, so joining two nodes may incur a one-to-all communications.

To tackle this problem, specified graph computing framework such as
Pregel~\cite{pregel},
GraphLab~\cite{graphlab}, PowerGraph~\cite{powergraph}, GraphX~\cite{graphx}
,GraphChi~\cite{graphchi} and so on~\cite{distributed:graphlab,kineograph}. These graph computing frameworks
adopt a vertex-edge computing model, and computations are modeled
as vertex and edge updates. For example, Pregel adopts the bulk
synchronous parallel (BSP) model, so each vertex updates its value according
to the edge and vertices (current vertex and neighboring vertices) values, and
the update algorithm.

However, the BSP model is found to be inefficient as the synchronous update
mechanism is always necessary for algorithms such as PageRank to
converge~\cite{graphlab,powergraph}. Moreover, multiple partition schemes (\eg{} vertex-centric~\cite{pregel},
edge-centric~\cite{xstream} and 3D partition~\cite{cube:osdi16}). The connectivity
of graph and the partition scheme can greatly affect the communication cost, and
further affect performance. PowerGraph~\cite{powergraph} is a advanced version of
GraphLab which takes the power law distribution of edges.

We have observed that most graph computing system optimize performance by
scheduling, new computing abstractions and partition schemes. The optimizations
may come from computation locality (\eg{} cpu-memory, data-partition),
selective scheduling and so on.

In this paper, we first analyze the basic components of a graph computing
system. Then, we give a survey of some popular graph computing system
such as Pregel, GraphLab, PowerGraph, GraphChi, GraphX and so on. We show
how these systems optimize their performance in terms of network traffic,
computing models.
We further evaluate the performance of several graph computing systems
and analyze their performance.
Our evaluations show that PowerGraph has the best performance in practice,
as it has optimized to work for graph with power law distribution.

The structure of this paper is as follows: \chref{pre} gives the definition
of graph computing system and its computing procedure. \chref{component}
introduce the components of graph computing systems. \chref{systems} gives
a survey and analysis of several graph computing system. \chref{eval} evaluated
system performance. \chref{con} gives the conclusion of the paper.

\section{Preliminary}
We first introduce some definition of graph computing.
Suppose there is a graph G = (V, E), $V$ and $E$ the vertex
and edges of the graph. Each vertex contains a state set $(s_1, s_2, ..., s_n)$,
and a set of neighboring vertices $v_{x1}, v_{x2}, ..., v{xn}$.
Each edge contains a state set (including the weight of edges) of
$(s_1, s_2, ..., s_n, w_i)$.

Initially, the graph G is stored in a database or files of HDFS. The system
first \textbf{load} data from storages. Each worker loads a subset of vertices
and edges of the graph G into memory according to the \textbf{partition scheme}
and initializes the values.
Then, each worker conducts iterative computations to
produce an output from vertex and edge values
according to the computing models. The output is used for updating
values in the second phase of computations. After that, the updated values
will be used in the new iteration.
When the computation converges, the system stops and the state of the graph
will be output to persistent storages.


\section{Graph Computing System}
\subsection{Computation Abstraction}
\subsection{Scheduling}
\subsection{Partition Scheme}


\section{Survey of Existing Systems}
\subsection{Pregel}
\subsection{GrabLab / PowerGraph}
\subsection{GraphX}
\subsection{ChiGraph}


\section{Evaluation}


\section{Conclusion}


%ACKNOWLEDGMENTS are optional
% \section{Acknowledgments}
% This section is optional; it is a location for you
% to acknowledge grants, funding, editing assistance and
% what have you.  In the present case, for example, the
% authors would like to thank Gerald Murray of ACM for
% his help in codifying this \textit{Author's Guide}
% and the \textbf{.cls} and \textbf{.tex} files that it describes.

\bibliographystyle{abbrv}
\bibliography{bib/biblio}  % sigproc-sp-csis8101.bib is the name of the Bibliography in this case

\end{document}

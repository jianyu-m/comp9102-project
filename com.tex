\section{Graph Computing System}
\subsection{Computation Abstraction}
The computation abstraction is the core of a graph computing system. The
most popular computation abstraction is BSP model and GAS model.
The advantage of graph computing systems lies in flexibility and performance.

For flexibility, the system should be able to compute on various
graph, even the topology of graph may be changing along with time.
Traditional MapReduce framework computes data in functional-like
APIs, and stateful computations are hard in practice. TODO.

For performance, as graph computing generally generate a tremendous
amount of network traffics for some operations,
it is highly desirable to reduce it. MapReduce framework needs to transfer
all states from previous computation to the next one, which causes high computation
costs. With graph computing, only a portion of states needs to be transfered.


\subsubsection{Bulk Synchronous Parallel Model}
The BSP model assumes that computation is done on multiple steps. A bulk contains
some subset of data, and a computation is done on these subset of data. After
computations, the output is used for synchronization.

For graph computing, Pregel makes the BSP model more specific.
Each vertex has two states: active and inactive, and computation
will only be conducted on active vertices. \fig{state} shows the
state machine of a vertex. Computations are done on multiple
supersteps. In the first superstep, all vertices is active and they
will vote to halt and convert to inactive. A vertex will be activate
when a message for the vertex is received.

In each superstep, a vertex get all messages sent to it and computes
result according to the current state of the vertex and sends messages
to its neighbors. \fig{bsp} shows the workflow of BSP.

A Combiner and Aggregator are used for transfering data for message
passing. The Combiner can combines states in each partitions to reduce
network traffics. For example, both vertex $b$ and $c$ send a same
message to vertex $a$, then this message can be combined into one.
On the other hand, Aggregator is used for aggregating output from multiple
vertex and even multiple iterations.

\subsubsection{GAS Model}
Gather, apply and scatter model (GAS) is more restricted than the BSP model.
Compared with BSP, a vertex in GAS can only interact with neighboring
vertices.

In \textbf{gather} phase, a vertex gathers information from all adjacent vertices, and
set a aggregate function for all the gathered information. Then, it \textbf{apply}
the function and updates values of vertices and edges. At last, in \textbf{scatter},
it can activate it neighboring vertices with information.


\subsection{Scheduling}
\subsection{Partition Scheme}

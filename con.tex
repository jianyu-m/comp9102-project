\section{Conclusion} \label{con}
In this paper, we have given a short summery of some popular graph computing
frameworks. We have conducted extensive evaluations for these frameworks
for analyze the performance gained in each techniques. Understanding
the effectiveness of each techniques and the design choices
make us better understand how to design a practical graph computing system.

We have observed that specified graph computing systems make optimizations
on scheduling, graph partition and caching and computation model.
Moreover, different datasets and algorithms can greatly affects performances.
For example, for graph with almost equally connected edges, a simple vertex cut
can performs well, while showing terrible performance for skewed graph.
At this scenario, an edge cut approach may be suitable.

When we summarize all these systems, we found that it is hard to get optimized
performance across all datasets and algorithms. It needs lots of techniques for
locality, scheduling and so on. It is challenging to design such a system. 

% Data locality
% Partition of dat
% scheduling (abstraction)

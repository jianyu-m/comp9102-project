\section{Survey of Existing Systems}
In this section, we will give a short survey of existing graph computing
systems. We will explain the techniques adopted in these system and
show some unique techniques for them.

\subsection{Pregel}
Pregel~\cite{pregel} is a C++ implementation of Google's graph computing system.
Pregel adopts the BSP computation abstract and uses the simple vertex cut
partition scheme. A $compute()$ function should be programmed for
graph computations on a vertex $v$. It accepts the messages from previous
supersteps and sends messages to next supersteps.

Pregel uses simple vertex cut for graph partition, this is practical
choice as a vertex in Pregel can send messages to arbitrary vertices
in a step. Therefore, communication patterns should be known for
a better partition of graph, which may not be possible in practice.
Pregel provides abstraction for changing topologies of graph.
It adopts two mechanisms: partial ordering and handlers, to guarantee
determinism of computation for changing of graphs.

In practice, a open-sourced implementation of google's Pregel is
Giraph~\cite{giraph}. It provides multi-threading programming and
memory optimization to get optimal performance.

\subsection{GraphLab / PowerGraph}
GraphLab~\cite{graphlab} chooses GAS computation model for graph
computing. With GAS, GraphLab open several optimizations
for performances. For example, it can separate computations
into multiple steps, and computations and message passing
can be working in the same time.

GraphLab also supports two kinds of execution: asynchronous
execution and synchronous execution.
For asynchronous execution, data is committed immediately
without communications to the neighboring vertices.
The change of value will only be visible in the next step
of computation. Although this approach can result in
inconsistency of data, it greatly improve performances
for algorithms such as PageRank. These algorithms
can incur large amounts of network traffics, but
values in consecutive iterations are really close.

For synchronous execution, it add a barrier in each
iteration for data exchanges by message passing.
This barrier guarantee consistency of values across
iterations, but the barrier can greatly degrade performance.

PowerGraph~\cite{powergraph} is a advance version of GraphLab.
It adopts vertex cut for graph partition and uses mirroring
of vertices to reduces network traffics for graph with power
law distribution. A network message are needed for data consistency of
vertices across nodes.

\subsection{GraphX}
GraphX~\cite{graphx} is a graph commuting framework based on Spark.
It introduces two RDD: VertexRDD and EdgeRDD for vertex and edge.
representations. Although GraphX is based on a general-purpose
dataflow commuting system, it adopted several optimizations for
graph computing.

GraphX adopts a Vertex Mirroring, Multicast Join and Incremental
View Maintenance
for network traffic optimizations.
Incremental View Maintenance uses to reduce unnecessary
move of unchanged data. As reported in the paper~\cite{graphx},
the performance of GraphX is close to GraphLab, an C++ implementation
system.

\subsection{GraphChi}
GraphChi~\cite{graphchi} is a graph computing system on a single machine.
It adopts Out-of-core computation for reducing accesses of second storages.
It also uses GAS model and proposes a selective scheduling scheme.
The selective scheduling scheme prioritizes computation with significant
values changing. The idea of this technique is that
most computations are not necessary for convergences of the whole graph,
most of the graph changes only a little or even does not change at all.
Therefore, this technique is good for convergences of the whole graph.
